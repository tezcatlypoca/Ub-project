\documentclass[a4paper]{article}
\usepackage{epsfig,epic,eepic,amssymb,version}
%%\graphicspath{{FIG/}{PS/}}
\usepackage[francais]{babel}
\usepackage[latin1]{inputenc}
\usepackage{amssymb}
\usepackage{moreverb}
\usepackage{listings}
\lstset{language=caml, extendedchars=true}
\newtheorem{theorem}{Theorem}
%%\newcommand\comment#1{}

%%\usepackage{graphicx}
\usepackage{amsfonts}
\def\C{\mathbb{C}}
\def\N{\mathbb{N}}
\def\Z{\mathbb{Z}}
\def\R{\mathbb{R}}

\begin{document}
\date{}
\title{TP : Plus longue s\'equence croissante}
%\author{D. Michelucci, E. Gavignet, L. Druoton, N.  Gastineau, Universit\'e de Dijon}
\maketitle

\begin{verbatim}
calcul commence
E =61  44  15  28  31  20  57  4   10  28
LT=1   1   1   2   3   2   4   1   2   3
n=10; plus longue sequence de longueur 4 :
E[6]=57;    E[4]=31;    E[3]=28;    E[2]=15; 
calcul a termine
quick commence
E =61  44  15  28  31  20  57  4   10  28
LT=1   1   1   2   3   2   4   1   2   3
V =-999 4   10  28  57  -999   -999   -999   -999   -999
n=10; plus longue sequence de longueur 4 :
E[6]=57;    E[4]=31;    E[3]=28;    E[2]=15; 
quick a termine
oui, les 2 methodes rendent le meme resultat avec n=10
********************************************************************************************
calcul commence
E =88  7  40  47  54  109 16  50  61 15 135 34 65 78 110 70 91 77 35 51
LT=1   1  2   3   4   5   2   4   5  2  6   3  6  7  8   7  8  8  4  5
n=20; plus longue sequence de longueur 8 :
E[17]=77;   E[15]=70;   E[12]=65;   E[8]=61;    E[7]=50;    E[3]=47;    E[2]=40;    E[1]=7; 
calcul a termine
quick commence
E =88  7 40 47 54 109 16 50 61 15 135 34 65 78 110 70 91 77 35 51
LT=1   1  2  3  4   5  2  4  5  2   6  3  6  7   8  7  8  8  4  5
V =-99 7  15 34 35  51 65 70 77 -   -  
n=20; plus longue sequence de longueur 8 :
E[17]=77;   E[15]=70;   E[12]=65;   E[8]=61;    E[7]=50;    E[3]=47;    E[2]=40;    E[1]=7; 
quick a termine
oui, les 2 methodes rendent le meme resultat avec n=20
\end{verbatim}
\end{document}
