\documentclass[a4paper]{article}
\usepackage{epsfig,epic,eepic,amssymb,version}
%%\graphicspath{{FIG/}{PS/}}
\usepackage[francais]{babel}
\usepackage[latin1]{inputenc}
\usepackage{amssymb}
\usepackage{moreverb}
\usepackage{listings}
\lstset{language=caml, extendedchars=true}
\newtheorem{theorem}{Theorem}
%%\newcommand\comment#1{}

%%\usepackage{graphicx}
\usepackage{amsfonts}
\def\C{\mathbb{C}}
\def\N{\mathbb{N}}
\def\Z{\mathbb{Z}}
\def\R{\mathbb{R}}

\begin{document}
\date{}
\title{TP : Plus longue s\'equence croissante}
%\author{D. Michelucci, E. Gavignet, L. Druoton, N.  Gastineau, Universit\'e de Dijon}
\maketitle

Votre programme et sa classe doivent porter votre nom (et pr\'enom en cas d'homonymie). Appelez l'encadrant chaque fois que vous avez termin\'e 
une question. 

Un tableau non tri\'e d'entiers naturels  $E[0], \ldots E[n-1]$  est donn\'e.
Il est g\'en\'er\'e par la m\'ethode \verb@ exemple@.
Le probl\`eme est de calculer la longueur de la s\'equence croissante (au sens large) la plus longue.
Dans cette s\'equence, tout \'el\'ement (sauf le dernier) 
est {\it inf\'erieur ou \'egal} \`a  son \'el\'ement suivant.
Par exemple, ci dessous, pour $$E=[61 ;   44 ;      {\bf 15} ;     {\bf  28} ;      {\bf 31} ;      20 ;      {\bf 57} ;      4 ;       10 ;      28]$$
la  s\'equence croissante la  plus longue a 4 \'el\'ements.
C'est~:
$$ E[2]=15; E[3]=28;        E[4]=31;        E[6]=57$$


\section*{Question 1. 6 points.}  Calculez   $LT[i]$,  la longueur 
de la s\'equence croissante la plus longue qui se termine en $E_i$ (et donc utilise $E_i$).  
Clairement $LT[0]=1$; ensuite  exprimez, pour l'indice $i$ croissant de 1 \`a $n-1$,  $LT[i]$ en fonction de $LT[0], \ldots LT[i-1]$;
avant de tenter de le programmer, faites le  "\`a la main" sur l'exemple ci-dessus, obtenu avec \verb@ exemple( 10)@.
Vous devez obtenir cette figure (\`a lire de gauche \`a droite)~:

\begin{comment}
$$
\begin{array}{|c|c|c|c|c|c|c|c|c|c|c|}
\hline
i &  0 &  1 &  2 &  3 &  4 &  5 &  6 &  7 &  8 &  9 \\
\hline
E_i & 61 &   44 &      15 &      28 &      31 &      20 &      57 &      4 &       10 &      28 \\
LT_i&1 &    1 &       1 &       2 &       3 &       2 &       4 &       1 &       2 &       3 \\
V_i & -- & 4 &       10 &      28 &      57 &      -- &    -- &    -- &    -- &    -- \\
\hline
\end{array}
$$
\end{comment}

\begin{center}
$$
\begin{array}{|c|c|c|c|c|c|c|c|c|c|c|}
\hline
i &  0 &  1 &  2 &  3 &  4 &  5 &  6 &  7 &  8 &  9 \\
\hline
E_i  & 61 &   44 &      15 &      28 &      31 &      20 &      57 &      4 &       10 &      28 \\
LT_i & 1 &    1 &       1 &       2 &       3 &       2 &       4 &       1 &       2 &       3 \\
\hline
\end{array}
$$
\end{center}

\section*{Question 2. 4 points.} Modifiez la m\'ethode d'affichage qui est fournie  pour que la s\'equence soit \'ecrite dans le bon sens
(croissant).  Par exemple, au lieu d'\'ecrire sur le terminal les termes de la s\'equence, vous les copiez dans un tableau auxiliaire, puis
vous parcourez ce tableau auxiliaire en ordre inverse et affichez ces \'el\'ements sur votre terminal.

\newpage
\section*{Question 3. 4 points.} Au lieu de chercher dans $E_0, \ldots E_{i-1}$ quelle est la plus longue s\'equence croissante que peut prolonger $E_i$,
cherchez la dans un tableau $V_1, V_2, \ldots V_L$~; $L=\max_{k=0}^{i-1} LT[k]$ est la longueur maximum des plus longues s\'equences connues quand $E_i$ est trait\'e, et
$V_l$ (avec $1\le l\le L$) est la valeur du dernier \'el\'ement (dans $E$) de la plus longue s\'equence croissante de longueur $l$.  
Si un tel $l$ n'existe pas (quand $E_i$ est plus petit que tous ses pr\'ec\'edents, ou que $i$ est nul),
alors votre recherche s\'equentielle dans $V_1, \ldots V_L$  rendra 0. 
Refaites l'exemple pr\'ec\'edent  \`a la main. Vous devez obtenir cette figure (\`a lire de haut en bas)~:

\begin{center}
$$
\begin{array}{|c|c|c||c|c|c|c||c|}
\hline
i & E_i & LT_i & V_1 & V_2 & V_3 & V_4 & L \\
\hline
0 & 61 & 1 & {\bf 61} & & & & 1 \\
1 & 44 & 1 & {\bf 44} & & & & 1 \\
2 & 15 & 1 & {\bf 15} & & & & 1 \\
3 & 28 & 2 & 15 & {\bf 28}  & & & 2 \\
4 & 31 & 3 & 15 & 28  & {\bf 31} & & 3 \\
5 & 20 & 2 & 15 & {\bf 20}  & 31 & & 3 \\
6 & 57 & 4 & 15 & 20  & 31 & {\bf 57} & 4 \\
7 & 4 & 1 & {\bf 4} & 20  & 31 & 57 & 4 \\
8 & 10 & 2 & 4 & {\bf 10}  & 31 & 57 & 4 \\
9 & 28 & 3 & 4 & 10  & {\bf 28} & 57 & 4 \\
\hline
\end{array}
$$
\end{center}

\section*{Question 4. 6 points.} Cherchez $l$ par dichotomie dans le tableau $V$.
Il faut trouver dans $V$ l'indice $l$ ($l$ est une longueur entre 1 et $L$) le plus grand tel que $V_l$ soit inf\'erieure ou \'egale \`a $E_i$.
Or le tableau $V_1, \ldots V_L$ est toujours  tri\'e par ordre croissant, comme sur l'exemple pr\'ec\'edent~:
$V_1\le  \ldots \le  V_L$.
V\'erifiez que toutes les  m\'ethodes rendent les m\^emes tableaux $LT$, pour des $E$ identiques, et que la seconde m\'ethode avec une recherche dichotomique est 
nette\-ment plus rapide pour des tailles de cent mille ou un million d'\'el\'ements dans $E$.

\section*{Bonus, 2 points.} (pour rattraper des points sur votre partiel, si vous avez les 20 points). La m\'ethode d'affichage est en $O(n)$, et 
on pr\'ef\'ererait une m\'ethode qui prenne un temps proportionnel \`a la longueur de la s\'equence. 
Dans la m\'ethode na\"ive, stockez dans un tableau $P_i$ l'indice $k$ de l'\'el\'ement $E_k$ pr\'ec\'edant
imm\'ediatement $E_i$ dans la plus longue s\'equence qui utilise $E_i$ (-1 s'il n'y en a pas de tel $k$). 
Ecrivez une m\'ethode d'affichage (en ordre inverse, si c'est le plus simple) qui utilise $P$. Remarque~: $P_i$ est similaire au pr\'ed\'ecesseur 
du sommet $i$
dans les m\'ethodes de plus courts chemins.

\section*{Bonus, 2 points.} Faites de m\^eme dans la m\'ethode rapide avec recherche dichotomique.

\end{document}
