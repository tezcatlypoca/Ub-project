\documentclass[a4paper]{article}
\usepackage{epsfig,epic,eepic,amssymb,version}
\graphicspath{{FIG/}{PS/}}
\usepackage[francais]{babel}
\usepackage[latin1]{inputenc}
\usepackage{amssymb}
\usepackage{moreverb}
\usepackage{listings}
\lstset{language=caml, extendedchars=true}
\newtheorem{theorem}{Theorem}
%%\newcommand\comment#1{}

\usepackage{graphicx}
\usepackage{amsfonts}
\def\C{\mathbb{C}}
\def\N{\mathbb{N}}
\def\Z{\mathbb{Z}}
\def\R{\mathbb{R}}

\begin{document}
\date{}
\title{TP : Plus longue s\'equence croissante}
\author{D. Michelucci, E. Gavignet, L. Druoton, N.  Gastineau, Universit\'e de Dijon}
\maketitle

Un tableau non tri\'e d'entiers naturels  $E[0], \ldots E[n-1]$  est donn\'e.
Il est g\'en\'er\'e par la m\'ethode \verb@ exemple@.
Le probl\`eme est de calculer la longueur de la s\'equence croissante la plus longue.
Dans cette s\'equence, tout \'el\'ement (sauf le dernier) 
est inf\'erieur ou \'egal \`a  son \'el\'ement suivant.
En premi\`ere approximation, vous pouvez supposer pour simplifier que tous les \'el\'ements sont diff\'erents.
Par exemple, ci dessous, pour $$E=[61 ;   44 ;      {\bf 15} ;     {\bf  28} ;      {\bf 31} ;      20 ;      {\bf 57} ;      4 ;       10 ;      28]$$
la  s\'equence croissante la  plus longue a 4 \'el\'ements.
C'est~:
$$ E[2]=15; E[3]=28;        E[4]=31;        E[6]=57$$


{\bf Question 1. 8 points.} Programmez la m\'ethode suivante.  D\'efinissez  $LT[i]$, comme \'etant la longueur 
de la s\'equence croissante la plus longue qui se termine (et utilise) $E_i$.  
$LT[0]=1$; ensuite  exprimez, pour l'indice $i$ croissant de 1 \`a $n-1$,  $LT[i]$ en fonction de $LT[0], \ldots LT[i-1]$;
avant de tenter de le programmer, faites le  "\`a la main" sur l'exemple ci-dessus~;
vous devez obtenir (ignorez la ligne $V$):

$$
\begin{array}{|c|c|c|c|c|c|c|c|c|c|c|}
\hline
i &  0 &  1 &  2 &  3 &  4 &  5 &  6 &  7 &  8 &  9 \\
\hline
E & 61 &   44 &      15 &      28 &      31 &      20 &      57 &      4 &       10 &      28 \\
LT&1 &    1 &       1 &       2 &       3 &       2 &       4 &       1 &       2 &       3 \\
V & -- & 4 &       10 &      28 &      57 &      -- &    -- &    -- &    -- &    -- \\
\hline
\end{array}
$$

L'exemple ci-dessus est obtenu avec \verb@ exemple( 10)@.


~ \\
{\bf Question 2. 4 points.} Modifiez la m\'ethode d'affichage qui est fournie  pour que la s\'equence soit \'ecrite dans le bon sens
(croissant).  Par exemple, au lieu d'\'ecrire sur le terminal les termes de la s\'equence, vous les copiez dans un tableau auxiliaire, puis
vous parcourez ce tableau auxiliaire en ordre inverse et imprimez ces \'el\'ements.

~ \\
{\bf Question 3 et 4. 2 points pour la question 3, 6 points pour la question 4.} 
La  m\'ethode na\"ive  est en temps $O(n^2)$. Programmez une m\'ethode en $O(n\log n)$. Pour cela,  stockez dans un tableau
$V[l]$ la derni\`ere valeur de la s\'equence de longueur $l$ (Attention, le tableau $V$ a un \'el\'ement de plus que $E$...). 
Il faut aussi g\'erer $L$, la plus grande 
longueur courante des s\'equences croissantes.
Quand vous cherchez quelle est la plus longue s\'equence croissante
que peut prolonger $E_i$, au lieu de tester $E_0, \ldots E_{i-1}$, vous chercherez, s\'equentiellement d'abord,  
par dichotomie ensuite, dans le tableau $V_1, \ldots V_L$ l'indice   $l$ ($l$ est une longueur, $1 \le l \le L$) le plus grand tel que $V_l\le E_i$; 
en effet $V$ est toujours tri\'e par ordre croissant~: $V_1\le V_2\le V_3\ldots$, etc~; faites \`a la main l'exemple ci-dessus pour vous
en rendre compte. Si un tel $l$ n'existe pas (quand $E[i]$ est plus petit que tous ses pr\'ec\'edents, ou que $i$ est nul),
alors votre recherche, s\'equentielle ou dichotomique, rendra 0. 
Attention~: n'oubliez pas de mettre \`a jour le tableau $V[ l] $ \`a chaque fois que vous trouvez une s\'equence de longueur $l$; autrement dit, apr\`es avoir 
calcul\'e $LT[i]$, il faut affecter~: $V[ LT[i] ]=E[i]$ (et \'eventuellement mettre \`a jour $L$).
V\'erifiez que les 2 m\'ethodes retournent les m\^emes tableaux $LT$, pour des $E$ identiques, et que la seconde m\'ethode avec une recherche dichotomique est 
nettement plus rapide pour des tailles de cent mille ou un million.

~ \\
{\bf Bonus, 2 points } (pour rattraper des points sur votre partiel, si vous avez les 20 points). La m\'ethode d'affichage est en $O(n)$, et 
on pr\'ef\'ererait une m\'ethode qui prenne un temps proportionnel \`a la longueur de la s\'equence. 
Dans la m\'ethode na\"ive, stockez dans un tableau $P_i$ l'indice $k$ de l'\'el\'ement $E_k$ pr\'ec\'edant
imm\'ediatement $E_i$ dans la plus longue s\'equence qui utilise $E_i$ (-1 s'il n'y en a pas de tel $k$). 
Ecrivez une m\'ethode d'affichage (en ordre inverse, si c'est le plus simple) qui utilise $P$. Remarque~: $P_i$ est similaire au pr\'ed\'ecesseur 
du sommet $i$
dans les m\'ethodes de plus court chemins.

~ \\
{\bf Bonus, 2 points } (pour rattraper des points sur votre partiel, si vous avez les 22 points). Faites de m\^eme dans la m\'ethode rapide.

\end{document}
