
\documentclass[a4paper]{article}
\usepackage{epsfig,epic,eepic,amssymb}
\graphicspath{{FIG/}{PS/}}
\usepackage[francais]{babel}
\usepackage[latin1]{inputenc}
\usepackage{amssymb}
\usepackage{moreverb}
\usepackage{listings}
\addtolength{\parskip}{\baselineskip}
\lstset{language=caml, extendedchars=true}
\newtheorem{theorem}{Theorem}

\usepackage{graphicx}
\usepackage{amsfonts}
\def\C{\mathbb{C}}
\def\N{\mathbb{N}}
\def\Z{\mathbb{Z}}
\def\R{\mathbb{R}}

\begin{document}
\title{TD. Structures de donn\'ees}
\author{Dominique Michelucci, Universit\'e de Dijon}
\maketitle

1. Programmer en Java la gestion d'une pile~: cr\'eer, empiler, d\'epiler, sommet. La pile est repr\'esent\'ee par un tableau. La taille du tableau sera multipli\'ee par 1.5 si n\'ecessaire.

2. Programmer en Java la gestion d'une pile~: cr\'eer, empiler, d\'epiler, sommet. La pile est repr\'esent\'ee par une liste.

3. Programmer en Java la gestion d'une file~: cr\'eer, ins\'erer, premier, supprimer. La file
est repr\'esent\'ee par un tableau. G\'erer l'agrandissement  quand c'est n\'ec\'essaire.

4. Programmer en Java la gestion d'une file~: cr\'eer, ins\'erer, premier, supprimer. La file
est repr\'esent\'ee par un couple de 2 listes, comme cela a \'et\'e expliqu\'e en cours.
Il faudra utiliser le retournement d'une liste en temps lin\'eaire.

5. Ecrire en Java la gestion d'un tas, repr\'esent\'e par un tableau~: cr\'eer, ins\'erer, minimum, supprimer (le minimum), modifier une valeur.
Les objets ins\'er\'es ont un num\'ero, de 0 \`a  $N-1$.
Il faut g\'erer une table $T[i]$ donnant l'indice, dans le tableau repr\'esentant le tas, de l'objet num\'ero $i$.
Vous programmerez ceci en TP, et vous l'utiliserez pour programmer la m\'ethode de Dijkstra. 

6. Ecrire en Java la gestion d'une table de hachage: cr\'eer, ins\'erer, retrouver, supprimer. La table sera  implant\'ee par un tableau de listes.
Vous supposerez que les cl\'es sont des cha\^ines de caract\`eres (String).
Rappel~: la m\'ethode String.hashCode() retourne une cl\'e de hachage.
Il faut pouvoir agrandir la table si n\'ec\'essaire.

\end{document}

