\documentclass[a4paper]{article}
\usepackage{epsfig,epic,eepic,amssymb}
\graphicspath{{FIG/}{PS/}}
\usepackage[francais]{babel}
\usepackage[latin1]{inputenc}
\usepackage{amssymb}
\usepackage{moreverb}
\usepackage{listings}
\lstset{language=caml, extendedchars=true}
\newtheorem{theorem}{Theorem}

\usepackage{graphicx}
\usepackage{amsfonts}
\def\C{\mathbb{C}}
\def\N{\mathbb{N}}
\def\Z{\mathbb{Z}}
\def\R{\mathbb{R}}

\begin{document}
\title{TD : Plus longue s\'equence croissante}
\author{Dominique Michelucci, Universit\'e de Dijon}
\maketitle

Un tableau non tri\'e d'entiers $E[0], \ldots E[n-1]$  est donn\'e.
Le probl\`eme est de calculer la longueur de la s\'equence croissante la plus longue.
Note: dans cette s\'equence, tout \'el\'ement (sauf le dernier) 
est inf\'erieur ou \'egal \`a  son \'el\'ement suivant.
Par exemple, si  $E=[0; 300; 100; 200; 1000; 400; 500;$ $ 1100; 900; 800; 600; 700; -100]$, alors
les s\'equences croissantes les plus longues ont 7 \'el\'ements.
L'une d'elles est $[0; 100; 200; 400; 500; 600; 700]$.

Proposez une m\'ethode en temps polynomial ($O(n^2)$). Par exemple, d\'efinir r\'ecursivement $LT[i]$, comme \'etant la longueur 
de la s\'equence croissante la plus longue qui se termine (et utilise) $E_i$. 
$LT[0]=1$. D\'efinissez $LT[i]$ en fonction de $LT[0], \ldots LT[i-1]$.
Exemple:
$$
\begin{array}{|c|c|c|c|c|c|c|c|c|c|c|c|c|c|}
\hline
i & 0 & 1 & 2 & 3 & 4 & 5 & 6 & 7 & 8 & 9 & 10 & 11 & 12  \\
\hline
E_i & 0 & 300 & 100 & 200 & 1000 & 400 & 500 & 1100 & 900 & 800 & 600 & 700 & -100 \\
\hline
LT_i & 1 & 2 & 2 & 3 & 4 & 4 & 5 & 6 & 6 & 6 & 6 &  7 & 1 \\
\hline
\end{array}
$$

Cette m\'ethode est en temps $O(n^2)$. Donnez une m\'ethode en $O(n\log n)$. Piste~: stockez dans un tableau
$V[l]$ la derni\`ere valeur de la s\'equence de longueur $l$. Quand vous cherchez quelle est la plus longue s\'equence croissante
que peut prolonger $E_i$, vous pouvez proc\'eder par dichotomie dans le tableau $V$. Il faut aussi g\'erer $L$, la plus grande 
longueur courante des s\'equences croissantes. N'oubliez pas de mettre \`a jour le tableau $V$.
Exemple: 
$$ 
\begin{array}{|c|c|c|c|c|c|c|c|c|c|c|c|c|c|}
\hline
i & 0 & 1 & 2 & 3 & 4 & 5 & 6 & 7 & 8 & 9 & 10 & 11 & 12  \\
\hline
E_i & 0 & 300 & 100 & 200 & 1000 & 400 & 500 & 1100 & 900 & 800 & 600 & 700 & -100 \\
\hline
LT_i & 1 & 2 & 2 & 3 & 4 & 4 & 5 & 6 & 6 & 6 & 6 & 7 & 1 \\
\hline
V_i & - & -100 & 100 & 200 & 400 & 500 & 600 & 700 & - & - & - & - & - \\
\hline
\end{array}
$$

La suite suppose pour simplifier que tous les \'el\'ements $E_i$ sont distincts.

Q: Soit $S_k$ la sous s\'equence form\'ee par les $E_i$ dont la LT (longueur terminale) vaut $k$. Que constatez-vous? Prouvez le.


Q. Soit $E_i$ le premier (i minimum) \'el\'ement tel que $LT[i]=l$, avec $l>1$.
Prouver que $E_i$ prolonge $E_k$, o\`u $k$ est l'\'el\'ement le plus \`a gauche ($k$ maximum, $k<i$) tel que $LT[k]=l-1$; autrement dit,
il est inutile de v\'erifier que $E[k] < E[i]$~: il l'est. 

Q: Quelle est la longueur de la s\'equence commune la plus longue entre une s\'equence croissante et une s\'equence d\'ecroissante~? (Tous les $E_i$ sont distincts)

Q: Soit $L$ la longueur de la s\'equence croissante la plus longue.

Q: quelle application de cette partition minimale en s\'equences d\'ecroissantes~?
En d\'eduire que toute partition en  s\'equences d\'ecroissantes a au minimum $L$ s\'equences d\'ecroissantes. 


Q: En d\'eduire une m\'ethode pour partitionner une s\'equence en un nombre minimum de s\'equences d\'ecroissantes.
\newpage

Q: Soit $S_k$ la sous s\'equence form\'ee par les $E_i$ dont la LT (longueur terminale) vaut $k$. Que constatez-vous? Prouvez le.

R: Elles sont d\'ecroissantes. Preuve triviale.


Q. Soit $E_i$ le premier (i minimum) \'el\'ement tel que $LT[i]=l$, avec $l>1$.
Prouver que $E_i$ prolonge $E_k$, o\`u $k$ est l'\'el\'ement le plus \`a gauche ($k$ maximum, $k<i$) tel que $LT[k]=l-1$; autrement dit,
il est inutile de v\'erifier que $E[k] < E[i]$~: il l'est. 

R: trivial, cf  Q pr\'ec\'edente.

Q: Quelle est la longueur de la s\'equence commune la plus longue entre une s\'equence croissante et une s\'equence d\'ecroissante~? (Tous les $E_i$ sont distincts)

R: la s\'equence commune la plus longue a soit 0 soit 1 seul \'el\'ement en commun. Si elle en a deux ($a, b$ dans cet ordre, ou plus), alors $a < b$  et $a > b$, contradiction.

Q: 
soit $L$ la longueur de la s\'equence croissante la plus longue.
En d\'eduire que toute partition en  s\'equences d\'ecroissantes a au minimum $K$ s\'equences d\'ecroissantes. 

R: 
soit $K$ la  s\'equence croissante la plus longue, de longueur $L$.
Soient $D_1, D_2, \ldots D_d$ une partition en  s\'equences d\'ecroissantes (il y a donc $d$  s\'equences d\'ecroissantes dans cette partition).
Tout \'el\'ement $E_i$ de $K$ se trouve dans exactement une seule des $D_1$ ou $D_2$ ou $\ldots D_d$.  Mais chaque $D_j$ ne peut contenir
qu'un seul des \'el\'ements de $K$. Donc $d\ge L$.


R: Soit $D_1, D_2, \ldots D_d$ une partition en  s\'equences d\'ecroissantes. On a vu que $d\ge L$. Il manque une preuve d'existence pour que le plus petit des $d$ 
soit vraiment \'egal \`a  $L$. Mais c'est donn\'e par une question pr\'ec\'edente. Donc: la longueur de la s\'equence croissante la plus longue est 
le nombre minimum de  s\'equences d\'ecroissantes des partitions en s\'equence d\'ecroissantes.

Q: En d\'eduire une m\'ethode pour partitionner une s\'equences en un ensemble de s\'equences d\'ecroissantes.

R: calculez la $LT_i$ maximum: $L$ par la m\'ethode en $O(n \log n)$. Soient $S_1, \ldots S_L$ ($S_k$ la sous s\'equence form\'ee par les $E_i$ dont la LT  vaut $k$): 
elles partitionnent la s\'equence en  $L$ s\'equences d\'ecroissantes.

Q: quelle application de cette partition minimale en s\'equences d\'ecroissantes~?

R: le tri par monotonie, pour des ensembles gigantesques qui ne rentrent pas en m\'emoire centrale; ils sont d'abord partitionn\'es en s\'equences
d\'ecroissantes; chacune est sauvegard\'e sur disque; ensuite ces fichiers tri\'es sont fusionn\'es~: c'est comme le tri par fusion mais 
il peut y avoir bien plus que 2 listes (ou fichiers) \`a fusionner; il faut parcourir s\'equentiellement ces fichiers; un tas (heap) est  utilis\'e
pour stocker les "t\^etes" de ces fichiers; quand la t\^ete du fichier $F_k$ est trait\'ee, il faut avancer la t\^ete de lecture dans le fichier $F_k$
et ins\'erer (ou remplacer) dans le tas. On esp\`ere que le tas rentre en m\'emoire centrale. Pour un ensemble al\'eatoire de $n$ \'el\'ements,
que vaut $L$~? Question math\'ematiquement non triviale, mais il est facile de mesurer. Sujet de TP possible...
\end{document}
