\documentclass[11pt]{article}
\usepackage[utf8]{inputenc}
\usepackage[french]{babel}
\usepackage{graphicx}
\usepackage[T1]{fontenc}
%\usepackage{amss}
\usepackage{amsmath}
\usepackage{amsfonts}
\usepackage{amssymb}

\newcommand\comment{}
 
\def\N{\mathbb N}
\def\R{\mathbb R}
\def\Q{\mathbb Q}
\def\Z{\mathbb Z}
\begin{document}
\title{EXAMEN D'ALGORITHMIQUE, I31A}
\date{}%{12-12-2013}
%\maketitle
{\bf\large EXAMEN D'ALGORITHMIQUE, I31A, 17-06-2014, L2} 

\medskip

{\it Tous vos documents sont autorisés, mais pas la copie de votre voisin.
N'écrivez aucun programme.
Ecrivez lisiblement.
Répondez aux questions dans l'ordre, en indiquant le numéro de chaque question.
Téléphone, calculette, tablette, ordinateur, lunettes Google sont interdits. 
Indication de longueur~: n'utilisez pas d'intercalaire.
Barème~: les 3 dernières questions sont notées sur 3 points chacunes, les autres sur 1 point chacune.
}

\newcommand\q[1]{\smallskip \noindent {\bf #1.}}
\medskip 

\q{1}  {D\'eroulez l'algorithme d'Euclide pour calculer le PGCD de 165 et 75}.
~

\q{2} {D\'eroulez l'algorithme d'Euclide \'etendu (ou Bézout) pour $a=165$ et $b=75$ en remplissant un tableau comme ci-dessous. {\it Dans la dernière ligne, $b$ est nul et $a$ est le PGCD cherché}.
L'algorithme vu en cours calcule $g$ le PGCD de $a$ et $b$, ainsi que $u$ et $v$. $u$ et $v$ sont tels que $au+bv=g=PGCD(a,b)$; on note $r=a \mod b$, et $q=\lfloor \frac{a}{b}\rfloor$
le quotient de $a$ par $b$.   Il n'y a qu'une seule réponse correcte.
{\large
$$
\begin{array}{|c|c|c|c|c|c|c|}
\hline
a\quad & b \quad& r \quad& q \quad& g \quad& u \quad& v \quad\\
\hline
\end{array}$$
}

\q{3} (suite) Soient $a\in \N$ et $b\in\N$ deux entiers naturels donnés. Soit une solution $(u, v)$, donnée elle aussi, 
du problème de Bézout: donc
$au + bv =\mbox{PGCD}(a, b)$. Mais cette solution $(u, v)$ n'est pas minimale~: 
il existe une autre solution $(u', v')$ telle que $au'+bv'=\mbox{PGCD}(a,b)$ et le vecteur $(u',v')$ est plus court que $(u,v)$. Proposez un algorithme
pour trouver une solution plus petite que $(u, v)$, s'il en existe. 
Attention~: vous devez utiliser $(u, v)$, pas seulement $a, b$.

\q{4} Vous avez 3 pièces d'or, une balance à deux plateaux, et vous savez qu'exactement une des pièces sur les trois est fausse~: elle est plus légère. Expliquez comment vous détectez la pièce fausse en une seule pesée.

\q{5} (suite)  Vous avez  $3^k$  pièces d'or, une balance à deux plateaux, et vous savez qu'exactement une des pièces  est fausse~: elle est plus légère. Expliquez comment vous détectez la pièce fausse en un minimum de pesées.

{
\q{6} {Quel est le nom de la méthode utilisée pour résoudre "le problème des reines" et "le compte est bon"~?}


\q{7} {Quel est l'ordre de grandeur du nombre minimum de 
produits matriciels nécessaires pour calculer $M^n$, où $M$ est une matrice carrée, et $n$ un entier naturel~?}

\q{8} {Proposez une m\'ethode rapide pour calculer $K_n$, o\`u $K_0, K_1$ sont donn\'es,
et $K_{n}= a_1 K_{n-1} + a_2 K_{n-2} + c$ quand $n>1$, avec des valeurs connues pour  $a_1, a_2, c$. Indication: utilisez une matrice de taille $3\times 3$ et l'algorithme de puissance rapide.}


\q{9} Les paysans russes réduisaient la multiplication  de deux entiers naturels à des sommes (dont des doublements), et des divisions par deux~: l'un des multiplicandes est implicitement décomposé en base 2. Donnez les formules récursives correspondantes pour $a\times b$~; vous supposerez que c'est $a$ qui est décomposé en base 2. 


\q{10} Définissez le problème SAT en 3 lignes au plus. Un algorithme rapide pour le résoudre est-il connu~?


\q{11} Un compilateur peut-il décider si deux fonctions sont équivalentes (deux fonctions sont équivalentes si elles donnent les mêmes résultats pour tous les arguments possibles)~?


\q{12} Soit l'équation $f(x)=x^2-4=0$. Définissez la fonction de Newton $N(x)$ associée. Quel 
sera le point fixe de $N$ en partant de $x_0=4$~? Pour $x\in [1/2, 5]$, dessinez la courbe $(x, y=N(x))$, la droite d'équation $x=y$,  et les premières étapes de la méthode de Newton, en partant de $x_0=4$, et de $x_0=1$. 
Pour cela vous calculerez (sans calculette) les valeurs numériques de $N(0), N(1), N(2), N(3), N(4), N(5)$.



\q{13} Un arbre binaire, représentant une expression arithmétique, est donné~; ses feuilles portent des nombres ou des noms 
de variables~; ses noeuds portent des noms d'opérations~: $+, \times$.  Décrivez un algorithme pour convertir 
cet arbre en une expression postfixe. Rappel~: $1, 2, 3, \times, +$ est une  expression postfixe, dont l'évaluation donne $1 + 2\times 3 =7$.

\q{14} (suite). Une expression postfixe est représentée par une liste d'éléments; chaque élément est soit un nombre, soit un nom de variable, soit un nom d'opération ($+, \times$). Décrivez un algorithme pour convertir cette expression postfixe en arbre binaire. Indication~: vous pouvez vous inspirer de l'algorithme d'évaluation d'une expression postfixe. 

\end{document}


\q{15} Les arcs d'un graphe orienté sont étiquetés par des probabilités de passage. Pour chaque sommet, la somme des probabilités des arcs 
qui en sortent vaut un.  Ce graphe est représenté par une matrice $P$, et $P_{l,c}$ (ou \verb@ P[l][c]@) est la probabilité de l'arc de $l$ vers $c$ quand il existe. $P_{l,c}$ est la probabilité d'être en $c$ à l'instant $t+1$ quand on se trouve en $l$  à l'instant $t$. Que vaut  $P_{l,c}$ quand il n'y a pas d'arc de $l$ vers $c$~? Indication~: considérez la question suivante.


\q{16} (suite)
$Q_{l,c}$ est la probabilité d'aller en {\it exactement} 2 arcs de $l$ à $c$, dans le graphe précédent, de matrice $P$. Exprimez $Q_{l,c}$ en fonction
de  $P$. Vous appelerez $n$ le nombre de sommets du graphe, que vous numéroterez de $1$ à $n$.




\q{17} (suite) Proposez un algorithme pour calculer le chemin le plus probable entre deux sommets donnés du graphe.



\q{18} Les arcs $l\rightarrow c$ d'un graphe orienté sont étiquetés par des probabilités de mourir lorsqu'on emprunte cet arc. 
Ce type de graphe est utilisé pour chercher les chemins les plus sûrs.
$M_{l,c}$ est la probabilité de mourir de l'arc $l\rightarrow c$ quand il existe. Que vaut $M_{l,c}$ quand l'arc n'existe pas~?
Indication~: considérez la question suivante.

\q{19}  (suite) Quelle est la probabilité de mourir lorsqu'on suit le chemin $a, b, c$ dans le graphe précédent~? Indication~: votre réponse doit dépendre de $M_{a,b}$ et 
$M_{b,c}$.


\q{20} (suite)  Proposez un algorithme pour calculer le chemin le plus sûr entre deux sommets donnés du graphe.  Le graphe est donné par sa matrice $M$.






\end{document}
