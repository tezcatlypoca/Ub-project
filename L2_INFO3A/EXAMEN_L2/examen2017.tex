\documentclass[a4paper]{article}
\usepackage[utf8]{inputenc}
%\usepackage[latin1]{inputenc}
\usepackage{epsfig,epic,eepic,amssymb}
\graphicspath{{FIG/}{PS/}}
\usepackage[francais]{babel}
\usepackage{amssymb}
\usepackage{moreverb}
\usepackage{listings}
\lstset{language=caml, extendedchars=true}
\newtheorem{theorem}{Theorem}

\setlength{\parindent}{0pt}
\setlength{\parskip}{2mm}


\usepackage{graphicx}
\usepackage{amsfonts}
\def\C{\mathbb{C}}
\def\N{\mathbb{N}}
\def\Z{\mathbb{Z}}
\def\R{\mathbb{R}}

\begin{document}
\title{EXAMEN 2017 INFO 31}
\date{?? 2017}
\maketitle
{\bf Répondez dans l'ordre aux questions. Répondez à chaque question en une ligne. Ecrivez lisiblement.}

Question 1. Calculez le PGCD $g$ et les coefficients de Bezout $u$ et $v$ de $a=84$ et $b=48$, avec le tableau habituel.
On rappelle que $au+bv=g$.
Dans la dernière ligne, $a=12$, $b=0$, $r$ et $q$ sont non définis, $g=12$, $u=1$ et $v=k$.
Seule est demandée la première ligne du tableau~:
$a=84$, $b=48$, $r=?$, $q=?$, $g=?$,  $u=?$, $v=?$. On rappelle que $q=a \div b$ et $r$ est $a$ modulo $b$.

Question 2. Quelle est la solution de l'équation~: $7^{\log_2 n}=n^x$. L'inconnue est $x$.

Question 1. Citer deux problèmes indécidables en informatique.


Question 2. Soient $(n_i, t_i=T(n_i))$ pour $i=1, \ldots N$. Les $n_i$ sont les
tailles des données de test pour un programme, et les $t_i$ sont les temps que met le programme pour calculer avec $n_i$ données.  Les points $(n_i, t_i)$ 
sont affichés dans un diagramme log-log~: autrement dit, les points
$(x_i=\log n_i, y_i=\log t_i)$ sont affichés.
Si le programme est en $O(n^d)$, sur quelle courbe se trouvent
les points $(x_i, y_i)$~?

Question 3 (suite). Même question si le programme est en $O(2^n)$~?

Question 4. Parfois, il est nécessaire de fusionner des entités dans la même classe d'équivalence. Quel est le nom anglais de ce problème~?

Question 5. Citer 3 méthodes pour calculer les plus courts chemins dans un graphe.

Question 6. Le produit de deux matrices $A$ (avec $l_A$ lignes de 0 à $l_A-1$ et
$c_A$ colonnes de 0 à $c_A-1$) et $B$ (avec $l_B$ lignes de 0 à $l_B-1$ et $c_B$ colonnes de 0 à $c_B-1$, avec $c_A=l_B$ est une matrice
$C$ de $l_A$ lignes et $c_B$ colonnes. Donnez la formule (donc ni un programme, ni un algorithme) définissant $C_{l, c}$ ($l$ est la ligne, $c$ est la colonne). 
%%$(AB)_{l, c}= \sum_{k=0}^{c_A-1} A_{l, k} B_{k, c}$.

Question 7 (suite).
Supposons que $A_{l, c}$ soit la distance entre l'élément numéro $l$ d'un ensemble  $E$ et
l'élément numéro $c$ d'un ensemble $F$. De même, $B_{l, c}$ est la distance entre l'élément numéro $l$ de $F$ et  l'élément numéro $c$ d'un ensemble $G$. 
Les ensembles $F, G, H$ sont disjoints deux à deux. Donnez en une formule la distance $C_{l, c}$ entre l'élément $l$ de $E$ et l'élément $c$ de $G$.
Vous pouvez noter $e, f, g$
les tailles de $E, F, G$.

Question 8 (suite). La formule précédente est vraie quand $E=F=G$. Il n'y a plus qu'une seule matrice, carrée, disons $A$, avec $A_{ii}=0$ (ce n'était pas vraie dans la question précédente).
Pensez à la puissance rapide d'une matrice. 
Quelle méthode cela suggère-t-il 
pour calculer les longueurs de tous les plus courts chemins dans le graphe~?

Question 9. Une suite $f_0, f_1, f_2\ldots$ est définie par $f_0$, $f_1$ et la relation $f_n=3 f_{n-1} - 2 f_{n-2}$ pour $n>1$. Quelle relation matricielle faut-il utiliser pour le calcul rapide de $f_n$~? 

Question 10. Existe-t-il une valeur de $f_0$ et $f_1$ telle que 
$f_n=2^n$~? Si oui lesquelles~?

Question. Strassen a trouvé un algorithme pour multiplier deux matrices carrées
$n$ par $n$ en temps $T(n)$ tel que~: $T(1)=1$, $T(n)=7 T(n/2)$. Quelle
est la complexité de l'algorithme~? Utiliser la notation $O(n^?)$. La preuve n'est pas demandée.

Question. Le chemin critique dans un graphe acyclique est-il un plus court chemin~?

Question 2. Un programme est en $O(n^2)$, avec $n$ la taille des données. Quand la taille des données est multipliée par 10, par combien est multiplié le temps de calcul~?


Question 4. Citer deux algorithmes de tri qui trient des entiers sans les comparer.

Question 5. En détaillant chaque phase, trier avec la méthode du tri par base (radix sort) les entiers~: 321, 331, 132, 123, 113, 231, 233, 212.

Question 5. Citer 4 structures de données (autres que les tableaux).

Question 5. Résoudre $n^x=3^{\log_2(n)}$. L'inconnue est $x$.

Question 6. Donnez un algorithme dont le temps de calcul $T(n)$ pour $n$ données
est solution de l'équation~: $T(1)=1, T(n)=2T(n/2)+n$.

Question 7. Quelle est la solution de l'équation~: $T(1)=1, T(n)=2T(n/2)+n$~? Utilisez la notation $O$.

Question 8. Citer deux problèmes résolus par programmation dynamique.

Question 9. Calculer le plus grand diviseur commun $g$ et les coefficients de Bezout $u$ et $v$ des deux entiers 70 et 49, c'est à dire
$70 u + 49v = g$. Utilisez une table avec les colonnes $a$, $b$, $r= a \mbox{ mod } b$, $q=a \div b$, $g$, $u$, $v$.

Question 10. Donnez d'autres coefficients de Bezout pour le problème précédent. Vous les exprimerez en fonction d'un entier relatif $t$.

%% Question 11. Même question que 9, mais utilisez des matrices de taille 2 par 2.

% Question 12. Quelle est la complexité de l'addition de deux matrices carrées de taille $n$ par $n$~? 


% Question 13. Quelle est la complexité (le nombre de multiplications de deux nombres) de l'algorithme évident de multiplication  de deux matrices carrées de taille $n$ par $n$~? $(AB)_{lc}=\sum_{k=1}^n A_{lk}B_{kc}$. 


% Question 14. Karatsuba a réduit le produit de deux nombres entiers de $n$ chiffres à trois produits de nombres de $n/2$ chiffres.  D'où une complexité~: $T(1)=1, T(n)=3 T(n/2) + n$. Résolvez~: $T(1)=1, T(n)=3 T(n/2)$.  Vous devez obtenir $T(n)=O(n^{??})$.




Question  15. Strassen a trouver une méthode de multiplication de deux matrices varrées de taille $n$ par $n$ qui nécessite
7 (et non 8) multiplications de matrices carrées de taille $n/2$ par $n/2$. Le temps de l'algorithme est donc
$T(1)=1, T(n)=7T(n/2)+n^2$. Le $n^2$ est dû au temps des additions et des soustractions de matrices. Résolvez, plus simplement~:
$T(1)=1, T(n)=7T(n/2)$. Vous devez obtenir $T(n)=O(n^?)$. A titre indicatif~: $\log_2(7)=ln(7)/ln(2)= 2.8073549$. 

\end{document}


