\documentclass[11pt]{article}
\usepackage[utf8]{inputenc}
\usepackage[french]{babel}
\usepackage{graphicx}
\usepackage[T1]{fontenc}
\usepackage{lmodern}
\usepackage{amsmath}
\usepackage{amsfonts}
\usepackage{amssymb}
\usepackage{ifthen}
\usepackage{multicol}
\usepackage{fancyhdr}
\newcommand{\marge}{18mm}
\usepackage[left=\marge,right=\marge,top=\marge,bottom=\marge]{geometry}
\pagestyle{fancy}
\setlength{\headheight}{14pt}
\chead{\ifthenelse{\thepage=1}{
 \textbf{Nom :}
  \makebox[12em]{\dotfill}
  \hspace{2em}
  \textbf{Pr\'enom :}
  \makebox[12em]{\dotfill}}{}}
\rfoot{\ifthenelse{\thepage=1}{\textit{(tournez la page s.v.p)}}{}}
\renewcommand{\headrulewidth}{0pt}
\linespread{1.3}
\setlength{\columnseprule}{0.2pt}

% Commandes sp\'ecifiques pour les QCM
\newboolean{correction}
% true pour afficher la correction
% false pour la masquer
\setboolean{correction}{false}
\newcounter{QNumber}
\newcommand{\Question}[2][:]{
 \stepcounter{QNumber}
  \noindent\textbf{Question \theQNumber} --
  #2~#1}
\newenvironment{Reponse}{
 \begin{list}{$\square$}{\leftmargin=4em}}{
 \end{list}\vspace{1em}}
\newcommand{\Vrai}{
 \item[\ifthenelse{\boolean{correction}}{$\blacksquare$}{$\square$}]}
\newcommand{\Faux}{\item[$\square$]}

\def\N{\mathbb N}
\def\R{\mathbb R}
\def\Q{\mathbb Q}
\def\Z{\mathbb Z}
\begin{document}
  \begin{center}
    \bfseries\LARGE
    \textit{INFO I31, partiel, 20 Nov 2012}\large
  \end{center}
  \sffamily
  \begin{itshape}
    Pour chaque question, r\'epondez directement sur la feuille.
Quand vous avez le choix, il n'y a qu'une seule bonne r\'eponse par question. Lisez et comprenez les questions avant d'y r\'epondre~!
  \end{itshape}
  \begin{multicols}{2}

\Question{D\'eroulez l'algorithme d'Euclide pour calculer le PGCD de 156 et 180 (en utilisant la division euclidienne, et pas la soustration)}
$$
\begin{array}{|c|c|c|}
\hline
\; a\; & \; b\; & \; r\; \\
\hline
180 & 156 & \quad \\
    &     & \\
    &     & \\
    &     & \\
    &     & \\
\hline
\end{array}$$


