\documentclass[a4paper,12pt]{article}
\usepackage[utf8]{inputenc}
%\usepackage[latin1]{inputenc}
\usepackage{epsfig,epic,eepic,amssymb}
\graphicspath{{FIG/}{PS/}}
\usepackage[francais]{babel}
\usepackage{amssymb}
\usepackage{moreverb}
\usepackage{listings}
\lstset{language=caml, extendedchars=true}
\newtheorem{theorem}{Theorem}

\setlength{\parindent}{0pt}
\setlength{\parskip}{2mm}


\usepackage{graphicx}
\usepackage{amsfonts}
\def\C{\mathbb{C}}
\def\N{\mathbb{N}}
\def\Z{\mathbb{Z}}
\def\R{\mathbb{R}}

\begin{document}
\title{PARTIEL 2017 INFO 31}
\date{24 octobre 2017}
\maketitle
{\bf Répondez dans l'ordre aux questions. Soyez concis et lisible.}

%Question 1. Citer deux problèmes indécidables en informatique.
Question 1. Quel problème a été résolu en cours par recherche arborescente ({\it  backtrack})~?


Question 2. Un programme est en $O(n^3)$, avec $n$ la taille des données. Quand la taille des données est multipliée par 10, par combien est multiplié le temps de calcul~?

Question 3. Un algorithme optimal de tri ordonne $n$ éléments, en les comparant. Combien fait-il de comparaisons~? Utilisez la notation $O$. 

Question 4. Citer deux algorithmes de tri qui trient des entiers sans les comparer.

%Question 5. En détaillant chaque phase, trier avec la méthode du tri par base (radix sort) les entiers~: 321, 331, 132, 123, 113, 231, 233, 212.

% Question 5. Citer 4 structures de données (autres que les tableaux).

Question 5. Résoudre $n^x=7^{\log_2(n)}$. L'inconnue est $x$.

Question 6. Donnez un algorithme dont le temps de calcul $T(n)$ pour $n$ données
est solution de l'équation~: $T(1)=1, T(n)=2T(n/2)+n$.

Question 7. Quelle est la solution de l'équation~: $T(1)=1, T(n)=2T(n/2)+n$~? Utilisez la notation $O$.

Question 8. Citer deux problèmes résolus par programmation dynamique.

Question 9. Calculer le plus grand diviseur commun $g$ et les coefficients de Bezout $u$ et $v$ des deux entiers 70 et 49, c'est à dire
$70 u + 49v = g$. Utilisez une table avec les colonnes $a$, $b$, $r= a \mbox{ mod } b$, $q=a \div b$, $g$, $u$, $v$.

Question 10. Donnez d'autres coefficients de Bezout pour le problème précédent. Vous les exprimerez en fonction d'un entier relatif $t$.

%% Question 11. Même question que 9, mais utilisez des matrices de taille 2 par 2.

% Question 12. Quelle est la complexité de l'addition de deux matrices carrées de taille $n$ par $n$~? 


% Question 13. Quelle est la complexité (le nombre de multiplications de deux nombres) de l'algorithme évident de multiplication  de deux matrices carrées de taille $n$ par $n$~? $(AB)_{lc}=\sum_{k=1}^n A_{lk}B_{kc}$. 


% Question 14. Karatsuba a réduit le produit de deux nombres entiers de $n$ chiffres à trois produits de nombres de $n/2$ chiffres.  D'où une complexité~: $T(1)=1, T(n)=3 T(n/2) + n$. Résolvez~: $T(1)=1, T(n)=3 T(n/2)$.  Vous devez obtenir $T(n)=O(n^{??})$.




%	Question  15. Strassen a trouver une méthode de multiplication de deux matrices varrées de taille $n$ par $n$ qui nécessite
%	7 (et non 8) multiplications de matrices carrées de taille $n/2$ par $n/2$. Le temps de l'algorithme est donc
%	$T(1)=1, T(n)=7T(n/2)+n^2$. Le $n^2$ est dû au temps des additions et des soustractions de matrices. Résolvez, plus simplement~:
%	$T(1)=1, T(n)=7T(n/2)$. Vous devez obtenir $T(n)=O(n^?)$. A titre indicatif~: $\log_2(7)=ln(7)/ln(2)= 2.8073549$. 

\end{document}


